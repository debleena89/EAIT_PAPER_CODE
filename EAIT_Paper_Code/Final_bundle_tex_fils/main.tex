\documentclass[runningheads,a4paper]{IEEEtran}
%\documentclass[conference]{IEEEtran}
\usepackage{graphicx}
\usepackage{amsmath}
\usepackage{comment}
\usepackage{amssymb}
\usepackage{xcolor}
\usepackage{caption}
\usepackage{url}
\usepackage[ruled]{algorithm2e}
\usepackage{algorithmic}
%\newtheorem{lemma}{Lemma}
%\newtheorem{theorem}{Theorem}
%\newtheorem{property}{Property}
%\newtheorem{corollary}{Corollary}
%\newtheorem{definition}{Definition}[section]
%\newtheorem{example}{Example}[section]

\title{ An automata theoretic framework for detecting schedulability attacks on cyber-physical systems}
%\author{}
%\institute{}
\newtheorem{definition}{Definition}

\author{\IEEEauthorblockN{\small{DEBLEENA DAS}\IEEEauthorrefmark{1},
\small{ANSUMAN BANERJEE}\IEEEauthorrefmark{2}, \small{SANDEEP K. SHUKLA}\IEEEauthorrefmark{3}}


\IEEEauthorblockA{\IEEEauthorrefmark{1}\IEEEauthorrefmark{3}\small{Department of Computer Science and Engineering,
Indian Institute of Technology, Kanpur}
\\\IEEEauthorrefmark{2}\small{Advanced Computing and Microelectronics Unit,
Indian Statistical Institute, Kolkata}\\
\small{Email: debleena@cse.iitk.ac.in}\IEEEauthorrefmark{1},
\small{ansuman@isical.ac.in}\IEEEauthorrefmark{2},
\small{sandeeps@cse.iitk.ac.in}\IEEEauthorrefmark{3}
}}

\begin{document}

\maketitle
\input{abstract}
\begin{keywords}
Cyber-physical systems, Network Security, B\"{u}chi automaton, Schdulability attack
\end{keywords}
\input{introduction}

\section{Related work}
\noindent
Weiss et al\cite{WeissA07}, depicts the idea of automata based control scheduling. They expressed
the communication interface among the control components by using formal language. In their
paper they have illustrated how the interfaces of discrete-time switched linear system
can be expressed using finite B\"{u}chi automaton. \cite{AlurW08} have described CPU scheduling
in terms of finite states over infinite words. This infinite words depicts the particular time 
slot for which a particular resource can be allotted to CPU. Weiss et al\cite{WeissFAA09} has also 
applied automata based scheduling on network, to formalize the effect of bus scheduling
and stability of the network. Further, \cite{GhoshMDHD16} describes how the interfacing among 
the control actions can be modeled by B\"{u}chi automaton to guarantee stable and reliable
scheduling. In this paper they have demonstrated, how to construct the scheduler automata 
where each state represents one particular control schedule and generated the permissible 
schedule in terms of $\omega$-regular language.


%\input{background}
\section{Problem statement and Motivating Example} \label{sec3}
\noindent

\textcolor{red}{To explain our problem we are considerig a system consisting of many control
applications. Each of these control applications is partitioned into a set of
tasks. Each of these tasks are mapped onto different processors. These tasks 
of the controllers communicates via shared buses. The tasks on each processors
are scheduling using a given scheduling policy. The messages on the shared buses
are also scheduled using a given arbitration policy.}

\textcolor{red}{ will use the term platform to refer to such an (i) architecture (ii) the 
different scheduling and arbitration policies used. }

\textcolor{red}{ let us consider two control applications control loop1 and control loop2.
Control loop1 performs two tasks $T_1$ and $T_2$, where $T_1$ and $T_2$ mapped onto 
different processors. $T_1$ for example can be a sensor task and $T_2$ can be a controller 
task. $T_1$ and $T_2$ communicate via a message stream M, consisting of an infinite sequence 
of message instances $m_1$, $m_2$, .... $T_3$ is another task from control loop2. Assuming 
that $T_1$, $T_2$, and $T_3$ senses at a constant periodic rate. The two control loops are 
reresented by finite state diagram as in Fig.1 }
 

\begin{figure}[h]
\begin{center}
\includegraphics[width=3.0in]{diagram.pdf}
\end{center}
\vspace{-0.1in}
\caption{{\em Control Loops sharing bus}}
\label{fig:automaton}
\end{figure}

In the above diagram, the controllers are represented by state diagram. 
There are 6 states in control loop1, among them only $q_2$ 
and $q_3$ are sharing the bus. In control loop2, there are 4 states, and among them
only one state $p_0$ is sharing the bus. So this common bus will be shared by only
these three states. All these three states will access the bus channel in Time-Division 
multiple access method under certain schedulable constraint. 

The bus access by these states can be thought of as an infinite tree, whose nodes 
are the states of the controlling automata which are sharing the bus and
each path of the tree gives a particular scheduling sequence. To model the problem,
we have considered a finite length tree say, length of the tree is 3. The leaf node of the tree
gives the possible scheduling sequences with these states.

\begin{figure}[h]
\begin{center}
\includegraphics[width=3.5in]{tree_automata.pdf}
\end{center}
\vspace{-0.1in}
\caption{{\em Tree automata for scheduling sequence }}
\label{fig:automaton}
\end{figure}

The scheduling sequence construct the states for the finite state machine. After adding
each participating states with the states of finite state machine, we want to get a cycle from
thses states. After getting the cycles we found that all of the participating states are
present in the omega regular expression extracted from this cycle.

\begin{figure}[h]
\begin{center}
\includegraphics[width=1.5in]{cycle_1.pdf}
\end{center}
\vspace{-0.1in}
\caption{{\em Cycle construction on FSM }}
\label{fig:automaton}
\end{figure}


The obtained omega Regular expression from the above transition is $(p_0q_2q_3)^\omega$.

Let us consider that control loop1 is replaced, $q_2$ and $q_3$ are not sharing the bus now.

\begin{figure}[h]
\begin{center}
\includegraphics[width=3.0in]{diagram_replaced.pdf}
\end{center}
\vspace{-0.1in}
\caption{{\em Control loop 1 got modified }}
\label{fig:automaton}
\end{figure}

In the above diagram, $q_2$ and $q_3$ from control loop1 are not sharing the bus, only $p_0$ is 
sharing the bus from control loop2. The tree automata only for $p_0$  state will look like:

\begin{figure}
\begin{center}
\includegraphics[width=5mm]{diagram_single_state.pdf}
\end{center}
\vspace{-0.1in}
\caption{{\em Tree automata for scheduling after control loop 1 got modified }}
\label{fig:automaton}
\end{figure}

In the above diagram, $q_2$ and $q_3$ from control loop1 are not sharing the bus, only
$p_0$ is sharing the bus from control loop2. After constructing the tree automata for
single state as depicted in Fig.5, we get a single sceduling sequence $p_0p_0p_0$ .
Only one transition is possible here:
The omega Regular expression obtained from this transition is $(p_0)^\omega$ . In this
expression we obtain only a single state $p_0$ , belongs to control loop2. But there
are no states from control loop1. In this situation we can say that code replacement attack
has taken place.

\begin{figure}[h]
\begin{center}
\includegraphics[width=1.0in]{one_state.pdf}
\end{center}
\vspace{-0.1in}
\caption{{\em Cycle construction on one state of FSM}}
\label{fig:automaton}
\end{figure}




%\section{Problem statement and Motivating Example} \label{sec3}
\noindent

\textcolor{red}{To explain our problem we are considerig a system consisting of many control
applications. Each of these control applications is partitioned into a set of
tasks. Each of these tasks are mapped onto different processors. These tasks 
of the controllers communicates via shared buses. The tasks on each processors
are scheduling using a given scheduling policy. The messages on the shared buses
are also scheduled using a given arbitration policy.}

\textcolor{red}{ will use the term platform to refer to such an (i) architecture (ii) the 
different scheduling and arbitration policies used. }

\textcolor{red}{ let us consider two control applications control loop1 and control loop2.
Control loop1 performs two tasks $T_1$ and $T_2$, where $T_1$ and $T_2$ mapped onto 
different processors. $T_1$ for example can be a sensor task and $T_2$ can be a controller 
task. $T_1$ and $T_2$ communicate via a message stream M, consisting of an infinite sequence 
of message instances $m_1$, $m_2$, .... $T_3$ is another task from control loop2. Assuming 
that $T_1$, $T_2$, and $T_3$ senses at a constant periodic rate. The two control loops are 
reresented by finite state diagram as in Fig.1 }
 

\begin{figure}[h]
\begin{center}
\includegraphics[width=3.0in]{diagram.pdf}
\end{center}
\vspace{-0.1in}
\caption{{\em Control Loops sharing bus}}
\label{fig:automaton}
\end{figure}

In the above diagram, the controllers are represented by state diagram. 
There are 6 states in control loop1, among them only $q_2$ 
and $q_3$ are sharing the bus. In control loop2, there are 4 states, and among them
only one state $p_0$ is sharing the bus. So this common bus will be shared by only
these three states. All these three states will access the bus channel in Time-Division 
multiple access method under certain schedulable constraint. 

The bus access by these states can be thought of as an infinite tree, whose nodes 
are the states of the controlling automata which are sharing the bus and
each path of the tree gives a particular scheduling sequence. To model the problem,
we have considered a finite length tree say, length of the tree is 3. The leaf node of the tree
gives the possible scheduling sequences with these states.

\begin{figure}[h]
\begin{center}
\includegraphics[width=3.5in]{tree_automata.pdf}
\end{center}
\vspace{-0.1in}
\caption{{\em Tree automata for scheduling sequence }}
\label{fig:automaton}
\end{figure}

The scheduling sequence construct the states for the finite state machine. After adding
each participating states with the states of finite state machine, we want to get a cycle from
thses states. After getting the cycles we found that all of the participating states are
present in the omega regular expression extracted from this cycle.

\begin{figure}[h]
\begin{center}
\includegraphics[width=1.5in]{cycle_1.pdf}
\end{center}
\vspace{-0.1in}
\caption{{\em Cycle construction on FSM }}
\label{fig:automaton}
\end{figure}


The obtained omega Regular expression from the above transition is $(p_0q_2q_3)^\omega$.

Let us consider that control loop1 is replaced, $q_2$ and $q_3$ are not sharing the bus now.

\begin{figure}[h]
\begin{center}
\includegraphics[width=3.0in]{diagram_replaced.pdf}
\end{center}
\vspace{-0.1in}
\caption{{\em Control loop 1 got modified }}
\label{fig:automaton}
\end{figure}

In the above diagram, $q_2$ and $q_3$ from control loop1 are not sharing the bus, only $p_0$ is 
sharing the bus from control loop2. The tree automata only for $p_0$  state will look like:

\begin{figure}
\begin{center}
\includegraphics[width=5mm]{diagram_single_state.pdf}
\end{center}
\vspace{-0.1in}
\caption{{\em Tree automata for scheduling after control loop 1 got modified }}
\label{fig:automaton}
\end{figure}

In the above diagram, $q_2$ and $q_3$ from control loop1 are not sharing the bus, only
$p_0$ is sharing the bus from control loop2. After constructing the tree automata for
single state as depicted in Fig.5, we get a single sceduling sequence $p_0p_0p_0$ .
Only one transition is possible here:
The omega Regular expression obtained from this transition is $(p_0)^\omega$ . In this
expression we obtain only a single state $p_0$ , belongs to control loop2. But there
are no states from control loop1. In this situation we can say that code replacement attack
has taken place.

\begin{figure}[h]
\begin{center}
\includegraphics[width=1.0in]{one_state.pdf}
\end{center}
\vspace{-0.1in}
\caption{{\em Cycle construction on one state of FSM}}
\label{fig:automaton}
\end{figure}




%\section{Problem statement and Motivating Example} \label{sec3}
\noindent
% A discrete time switched linear system is modeled by a finite set of real 
% $n \times n$ matrices $ \sum = \{A_i\}_{i \in I}$, $|I|$ $<$ $\infty $, as defined in~\cite{AlurW08}. 
% Infinite words $\sigma = I^\omega$ are used to describe control schedules. 
% Given a schedule $\sigma \in I^\omega$, and an initial state $x_0$ $\in$ 
% $\mathbb{R}^n$, the dynamics of the system is given by:
% 
% \begin{center}
%        \textcolor{black}{ $x_{k+1} = A_{\sigma_k} x_k$ }
% \end{center}
%\textbf{Problem definition of the problem:}
For a discrete-time  linear time invariant system as above, we assume we are given 
a set of $n$ controllers $ A = { {A_1}, {A_2}, ....{A_n}} $, each represented 
with the dynamics of its control actions specified with individual transition matrices. As discussed before,  
a control schedule is modeled as an interleaving of these control actions with the plant dynamics, as specified 
in the transition matrices. A control schedule is considered exponentially stable, if the composition of the plant and the controllers entail a non-empty language. 
As discussed in the previous section, the language of exponential stability is 
B\"{u}chi-recognizable. Formally, we are given the following:

\begin{itemize}
 \item We are given a plant ${\cal P}$, specified, as in Section~\ref{sec2}.
 \item We have  a set of $n$ control actions $A_1, A_2, \ldots A_n$ for $n$ controllers.
 \item A stability computation window length $L$ and a stability margin 
$\rho$ for $(L, \rho)$ exponential stability.
 \item A modified set of controllers $A'_1, A'_2, \ldots A'_n$, possibly resulting out of code replacement attacks. The two sets $A_1, A_2, \ldots A_n$ 
       and $A'_1, A'_2, \ldots A'_n$ differ in at least one $i$ (1 $\leq$ $i$ $\leq$ n).
%  \item For $L$ length scheduling sequence there will be possible $ n^L $ combinations.
%  \item Only exponentially stable sequence will be considered. Exponential stability$(L,\rho)$ 
%        computed for $L$ length sequence.
%  \item To construct a B\"{u}chi automaton with these stable states.   
\end{itemize}

\noindent
{\bf Objective:} 
Our objective is to develop an automatic framework for analyzing the effect of the modified set of controllers on the schedulability of the system as far as the 
exponential stability criterion is considered.
We assume that the original system is schedulable, and there indeed exists a scheduler that can ensure that the system resulting out of the composition of the 
plant and the original set of controllers is exponentially stable. 
 If the system resulting out of 
the combination of the plant and the modified set of controllers does not 
remain schedulable, we conclude that the system has been subject to a code replacement attack. We follow the same prescription for schedulability analysis, as given in~\cite{GhoshMDHD16}. %for schedulability analysis.    
%        If a B\"{u}chi Automaton cannot be constructed with these stable states, 
%        then it will be concluded that at least one of the control actions has been 
%        attacked, otherwise these stable states would \textcolor{red}{have Resulted} into a B\"{u}chi Automaton
%        which possess at least one cycle over the states.
The following example illustrates the philosophy. 
     
We consider a discrete-time switched linear system having
three control actions ${A_0, A_1, A_2}$ each specified by $2$ $\times$ $2$ matrices. Let us consider the stability computation window length is $4$, i.e., $\sigma = { \sigma_1, \sigma_2, 
\sigma_3, \sigma_4}$. So possible number of schedules for this scenario is $3^4$ = $81$.  
But all these schedules do not result into an exponentially stable state. 
% From all 
% possible states some states will be filtered out to guarantee exponential stability, 
% say $z$ number of states are stable among these $81$ states. Now, adding all possible
% transitions on these $z$ states will get a finite state machine. If a cycle can be 
% obtained from this state transitions, it will signify that this resulting automaton
% is a B\"{u}chi automaton and the string obtained from the cycle is $\omega$-regular. 
% A schedulable sequences of control actions is a $\omega$-regular string which 
% guarantees that the chosen set of schedules for the control actions.
% 
% 
% \noindent
% The above idea is captured to detect code replacement attack on such a 
% discrete-time switched linear system. In this work we have tried to illustrate an idea
% that if a control action is hijacked then the system will not remain schedulable, i.e.,  
% the states cannot form B\"{u}chi automaton.
Let us consider the following three control actions:

   $A_0 = ([2.0, -1.75],[2.0, -2.0])$
   
   $A_1 = ([0.25, 1.75],[0.25, -0.25])$

   $A_2 = ([0.5, 1.85],[0.5, -0.5])$
   
\noindent
Assuming a stability margin of 0.5, we can easily verify that among the possible $81$ states, only $14$ states are stable, and the automaton formed
with these states contains cycles. This means the control actions are schedulable.

%\vspace{-0.1 in}
\begin{figure}[h]
\begin{center}
\includegraphics[width=90mm, height=20mm]{schedulable_new.pdf}
\end{center}
%\vspace{-0.1in}
\caption{{\em Automaton construction with stable states}}
\label{fig:cycle found}
\end{figure}


\noindent
We now consider another set of control actions:
   
   $A_0 = ([2.0, -1.75],[2.0, -2.0])$
   
   $A_1 = ([1.65, -2.5],[1.66, 1.0])$

   $A_2 = ([0.5, 1.85],[0.5, -0.5])$
   
%\vspace{-0.1 in}
\begin{figure}

\begin{center}
\includegraphics[width=10mm, height = 10mm]{non_schedulable.pdf}
\end{center}
%\vspace{-0.1in}
\caption{{\em  Only one state found}}
\label{fig:Only one state}
\end{figure}

\noindent
In this case, among the possible $81$ states, only $1$ state is stable. A cycle cannot be obtained 
from this state, which results in a non-schedulable control action.
In the modified controller set, only $A_1$ has been changed from the previous 
set of control actions, the other 
two control actions remain the same. As we see from above, the loss of exponential stability can be a useful indicator for concluding that code replacement attacks have been carried out on a given system. This is the problem setting addressed in this paper.


% \subsection{Future Scope}
% \noindent
% In lieu of the scalability issue with model based assertion verification methods, dynamic assertion based verification techniques are inevitable to follow for software verification sign-off. Hence, we see to extend the proposed frame work to employ a mechanism for the generation of minimal subset of test suites that cover all the user specified assertions and subsequently use this framework to complete the testing process. 
%\vspace{-0.2 in}




\noindent
\\\section{Solution Architecture} \label{sec4}
\noindent
 
%\includegraphics[width=3.0in]{tree_space.pdf}

\textbf{ Approaches to solve the above stated Problem:}

  There are $n$ number of control actions. We are considering that each control application
  can be represented by a B\"{u}chi Automaton. In this B\"{u}chi Automaton the accepting states
  are the bus accessing states of that automata. At least one states from each of them 
  are sharing a common bus. In this paper, we are proposing an idea to   detect 
  code replacement attack. 
  After a particular interval of time the system will check the schedulability 
  of the system to assure that code replacement attack has not taken place. The main
  idea behind the algorithm is to check that the resulting automata after the intersection
  is also a B\"{u}chi Automaton.
  The following algorithm represents the method to solve the problem. Before starting
  the algorithm we are considering there are $n$ number of control loops.
  
\begin{algorithm}
  \caption{ Code Replace attack detection}
  \label{Code Replace attack detection}
  \begin{algorithmic}[1]
  
  \STATE Cross product is calculated on the given $n$ number of automata.
  \STATE Starting from the initial state, only reachable states are considered to get the product automata.
  \STATE From this product automata, applying the procedure as stated in the Background section, we get the
         resulting intersection automata for the given B\"{u}chi Automaton.
  \STATE Next task is to find out at least one cycle from this resulting intersection automata, the run of the
  cycle should comprise of the states containig the acceting states from each control loop.
  \STATE If such a cycle is not obtained we can conclude that a code replacement attack has taken place.
  \end{algorithmic}

\end{algorithm}


   
\begin{figure}
\begin{center}
%\includegraphics[width=75mm]{algorithn.pdf}
\end{center}
%\vspace{-0.1in}
%\caption{{\em Flow chart of our idea}}
\label{fig:Algorithm}
\end{figure}

   
%\textbf{Construction of B\"{u}chi automaton:}

%B\"{u}chi Automaton is constructed over the leaf nodes of a tree with $L$ depths and 
%each node contains $n$ number of 
%children. So each leaf represents a stable combination of $L$ length control 
%actions. 

%     An automaton or a finite state machine is considered as B\"{u}chi automaton when there exist 
%at least one cycle over the states where there is at least one accepting or final state. 
%But in this case, all of the stable states are final states so if we find at least one cycle
%over these states, we can say that it is a B\"{u}chi automaton.

%The automaton can be described as follows:
%$G = \langle S,I,M,T,{s_0} \rangle $

%\begin{itemize}
% \item Here $S$ is set of states, which is $L$ length sequence of the control actions
% \item $I$ is possible inputs, here it is set of control actions i.e $\{ A_0, A_1, ...\} $
% \item $M$ represents the transition functions, which states that $ M: S$ x $I \rightarrow S$
% \item $T$ is set of final states, here each state is final state because all of them are
       %exponentially stable.
 %\item ${s_0}$ is initial state
%\end{itemize}

%Here, each state which is the $L$ length sequences of the control actions, can take one of the 
%control actions as input and switch to another exponentially stable state. If the possible next 
%states are not exponentially stable then that transition will not be permissible. 
%The control action is appended at the  \textcolor{red}{leftmost} side of the state and to make it again of $L$ 
%length, one sequence from  \textcolor{red}{rightmost} side is discarded. Now this new state is compared 
%with all of the exponentially stable states including itself, if any match found, then this 
%transition is accepted from the previous state for this control action.


%Consider these following four states :

%  $A_0A_0A_1A_1$    $A_0A_1A_1A_0$    $A_1A_1A_0A_0$     $A_1A_0A_0A_1$     
  
  
% \begin{figure}[h]
%\begin{center}
%\includegraphics[width=3.0in]{state_change_diagram.pdf}
%\end{center}
%\vspace{-0.1in}
%\caption{{\em State transition of the automaton}}
%\label{fig:automaton}
%\end{figure}
   
%   Now the first state $A_0A_0A_1A_1$ when take $A_0$ as a input string then it 
%   becomes $A_0A_1A_1A_0$ which is the second state, first state with $A_1$ and $A_2$
%   are not stable states so are not present in the list. This second state with input 
%   state $A_0$ goes to the third state and the third state with input string $A_1$ goes 
%   to the fourth state. This is represented in the above diagram.




%\section{Toolflow and case study} \label{sec5}
\noindent
We have built an end-to-end tool in Python for code replacement attack detection.
The tool takes in a set of control loop descriptions, computes their intersection and 
implements the cycle detection step. The tool has been applied to a number of small 
examples of synthetic control applications and their variants. The time taken for
the tool to run on a QuadCore Intel machine is in the order of milliseconds, and the
peak memory consumption is of the order of Kilobytes. Due to the lack of standard open
source benchmarks in this domain, we have not been yet able to check the performance of
our tool on more non-trivial benchmarks. We describe below one case study that was analyzed 
successfully by our tool. An overview of our toolflow is shown in Fig\ref{tool_algorithm}.

\begin{figure}
\begin{center}
\includegraphics[width=50mm]{algorithm.pdf}
\end{center}
%\vspace{-0.1in}
\caption{{\em Sample automata}}
\label{tool_algorithm}
\end{figure}


In this section, we have described some examples which depicts the difference of intersection
automaton before and after the code replacement. 
Let us consider three control automata as shown in Fig.\ref{automata_orginal}

\begin{figure}
\begin{center}
\includegraphics[width=50mm]{Automaton_original.pdf}
\end{center}
%\vspace{-0.1in}
\caption{{\em Sample automata}}
\label{automata_orginal}
\end{figure}

The intersection automaton of the given automata will look like Fig.\ref{graph_orginal}

\begin{figure}
\begin{center}
\includegraphics[width=50mm]{graph_original.eps}
\end{center}
%\vspace{-0.1in}
\caption{{\em Intersection automata of $P, Q$ and $R$ before replacement}}
\label{graph_orginal}
\end{figure}

Now we will see  what happens after each of the automaton get replaced.

After P automaton replacement the automaton representing the control 
loops will look like Fig. \ref{automata_p_replaced}

\begin{figure}
\begin{center}
\includegraphics[width=50mm]{graph_after_p_replacement.pdf}
\end{center}
%\vspace{-0.1in}
\caption{{\em sample automata: P automaton replaced}}
\label{automata_p_replaced}
\end{figure}


The intersection automaton will look after P automaton replacement Fig.\ref{graph_p_replaced}

\begin{figure}
\begin{center}
\includegraphics[width=50mm]{graph_p_replaced.eps}
\end{center}
%\vspace{-0.1in}
\caption{{\em Intersection automata of $P, Q and R$ after $P$ got replaced}}
\label{graph_p_replaced}
\end{figure}

After Q automaton replacement the control loops will look like Fig. \ref{automata_q_replaced}

\begin{figure}
\begin{center}
\includegraphics[width=50mm]{graph_after_q_replacement.pdf}
\end{center}
%\vspace{-0.1in}
\caption{{\em  sample automata: Q automaton replaced}}
\label{automata_q_replaced}
\end{figure}



The intersection automaton will look after Q automaton replacement Fig.\ref{graph_q_replaced}

 \begin{figure}
\begin{center}
\includegraphics[width=50mm]{graph_q_replaced.eps}
\end{center}
%\vspace{-0.1in}
\caption{{\em Intersection automata of $P, Q and R$ after $Q$ got replaced}}
\label{graph_q_replaced}
\end{figure}


After R automaton replacement the control loops will look like Fig. \ref{automata_r_replaced}

\begin{figure}
\begin{center}
\includegraphics[width=50mm]{graph_after_r_replacement.pdf}
\end{center}
%\vspace{-0.1in}
\caption{{\em sample automata: R automaton replaced}}
\label{automata_r_replaced}
\end{figure}


The intersection automaton will look after R automaton replacement Fig.\ref{graph_r_replaced}

  \begin{figure}
\begin{center}
\includegraphics[width=50mm]{graph_r_replaced.eps}
\end{center}
%\vspace{-0.1in}
\caption{{\em Intersection automata of $P, Q and R$ after $R$ got replaced}}
\label{graph_r_replaced}
\end{figure}
\input{Result}
%\section{Experiment} \label{sec6}
\noindent

\section{Conclusion and Future Work} \label{sec7}
\noindent

\section{Acknowledgement} \label{sec8}
\noindent
This work was supported partially by the USAFOSR (OAD) project fund (FA2386-16-1-4099).
\bibliographystyle{IEEEtran}
{\small
\nocite{*}
\bibliography{references}
}
%\bibliographystyle{abbrv}
%\bibliography{main}
\end{document}
