\noindent
\\\section{Solution Architecture} \label{sec4}
\noindent
 
%\includegraphics[width=3.0in]{tree_space.pdf}

\textbf{ Approaches to solve the above stated Problem:}

  There are $n$ number of control actions. We are considering that each control application
  can be represented by a B\"{u}chi Automaton. In this B\"{u}chi Automaton the accepting states
  are the bus accessing states of that automata. At least one states from each of them 
  are sharing a common bus. In this paper, we are proposing an idea to   detect 
  code replacement attack. 
  After a particular interval of time the system will check the schedulability 
  of the system to assure that code replacement attack has not taken place. The main
  idea behind the algorithm is to check that the resulting automata after the intersection
  is also a B\"{u}chi Automaton.
  The following algorithm represents the method to solve the problem. Before starting
  the algorithm we are considering there are $n$ number of control loops.
  
\begin{algorithm}
  \caption{ Code Replace attack detection}
  \label{Code Replace attack detection}
  \begin{algorithmic}[1]
  
  \STATE Cross product is calculated on the given $n$ number of automata.
  \STATE Starting from the initial state, only reachable states are considered to get the product automata.
  \STATE From this product automata, applying the procedure as stated in the Background section, we get the
         resulting intersection automata for the given B\"{u}chi Automaton.
  \STATE Next task is to find out at least one cycle from this resulting intersection automata, the run of the
  cycle should comprise of the states containig the acceting states from each control loop.
  \STATE If such a cycle is not obtained we can conclude that a code replacement attack has taken place.
  \end{algorithmic}

\end{algorithm}


   
\begin{figure}
\begin{center}
%\includegraphics[width=75mm]{algorithn.pdf}
\end{center}
%\vspace{-0.1in}
%\caption{{\em Flow chart of our idea}}
\label{fig:Algorithm}
\end{figure}

   
%\textbf{Construction of B\"{u}chi automaton:}

%B\"{u}chi Automaton is constructed over the leaf nodes of a tree with $L$ depths and 
%each node contains $n$ number of 
%children. So each leaf represents a stable combination of $L$ length control 
%actions. 

%     An automaton or a finite state machine is considered as B\"{u}chi automaton when there exist 
%at least one cycle over the states where there is at least one accepting or final state. 
%But in this case, all of the stable states are final states so if we find at least one cycle
%over these states, we can say that it is a B\"{u}chi automaton.

%The automaton can be described as follows:
%$G = \langle S,I,M,T,{s_0} \rangle $

%\begin{itemize}
% \item Here $S$ is set of states, which is $L$ length sequence of the control actions
% \item $I$ is possible inputs, here it is set of control actions i.e $\{ A_0, A_1, ...\} $
% \item $M$ represents the transition functions, which states that $ M: S$ x $I \rightarrow S$
% \item $T$ is set of final states, here each state is final state because all of them are
       %exponentially stable.
 %\item ${s_0}$ is initial state
%\end{itemize}

%Here, each state which is the $L$ length sequences of the control actions, can take one of the 
%control actions as input and switch to another exponentially stable state. If the possible next 
%states are not exponentially stable then that transition will not be permissible. 
%The control action is appended at the  \textcolor{red}{leftmost} side of the state and to make it again of $L$ 
%length, one sequence from  \textcolor{red}{rightmost} side is discarded. Now this new state is compared 
%with all of the exponentially stable states including itself, if any match found, then this 
%transition is accepted from the previous state for this control action.


%Consider these following four states :

%  $A_0A_0A_1A_1$    $A_0A_1A_1A_0$    $A_1A_1A_0A_0$     $A_1A_0A_0A_1$     
  
  
% \begin{figure}[h]
%\begin{center}
%\includegraphics[width=3.0in]{state_change_diagram.pdf}
%\end{center}
%\vspace{-0.1in}
%\caption{{\em State transition of the automaton}}
%\label{fig:automaton}
%\end{figure}
   
%   Now the first state $A_0A_0A_1A_1$ when take $A_0$ as a input string then it 
%   becomes $A_0A_1A_1A_0$ which is the second state, first state with $A_1$ and $A_2$
%   are not stable states so are not present in the list. This second state with input 
%   state $A_0$ goes to the third state and the third state with input string $A_1$ goes 
%   to the fourth state. This is represented in the above diagram.



