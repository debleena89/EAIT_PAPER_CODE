\section{Related work}
\noindent
Weiss et al\cite{WeissA07}, depicts the idea of automata based control scheduling. They expressed
the communication interface among the control components by using formal language. In their
paper they have illustrated how the interfaces of discrete-time switched linear system
can be expressed using finite B\"{u}chi automaton. \cite{AlurW08} have described CPU scheduling
in terms of finite states over infinite words. This infinite words depicts the particular time 
slot for which a particular resource can be allotted to CPU. Weiss et al\cite{WeissFAA09} has also 
applied automata based scheduling on network, to formalize the effect of bus scheduling
and stability of the network. Further, \cite{GhoshMDHD16} describes how the interfacing among 
the control actions can be modeled by B\"{u}chi automaton to guarantee stable and reliable
scheduling. In this paper they have demonstrated, how to construct the scheduler automata 
where each state represents one particular control schedule and generated the permissible 
schedule in terms of $\omega$-regular language.

