\section{Toolflow and case study} \label{sec5}
\noindent
We have built an end-to-end tool in Python for code replacement attack detection.
The tool takes in a set of control loop descriptions, computes their intersection and 
implements the cycle detection step. The tool has been applied to a number of small 
examples of synthetic control applications and their variants. The time taken for
the tool to run on a QuadCore Intel machine is in the order of milliseconds, and the
peak memory consumption is of the order of Kilobytes. Due to the lack of standard open
source benchmarks in this domain, we have not been yet able to check the performance of
our tool on more non-trivial benchmarks. We describe below one case study that was analyzed 
successfully by our tool. An overview of our toolflow is shown in Fig\ref{tool_algorithm}.

\begin{figure}
\begin{center}
\includegraphics[width=50mm]{algorithm.pdf}
\end{center}
%\vspace{-0.1in}
\caption{{\em Sample automata}}
\label{tool_algorithm}
\end{figure}


In this section, we have described some examples which depicts the difference of intersection
automaton before and after the code replacement. 
Let us consider three control automata as shown in Fig.\ref{automata_orginal}

\begin{figure}
\begin{center}
\includegraphics[width=50mm]{Automaton_original.pdf}
\end{center}
%\vspace{-0.1in}
\caption{{\em Sample automata}}
\label{automata_orginal}
\end{figure}

The intersection automaton of the given automata will look like Fig.\ref{graph_orginal}

\begin{figure}
\begin{center}
\includegraphics[width=50mm]{graph_original.eps}
\end{center}
%\vspace{-0.1in}
\caption{{\em Intersection automata of $P, Q$ and $R$ before replacement}}
\label{graph_orginal}
\end{figure}

Now we will see  what happens after each of the automaton get replaced.

After P automaton replacement the automaton representing the control 
loops will look like Fig. \ref{automata_p_replaced}

\begin{figure}
\begin{center}
\includegraphics[width=50mm]{graph_after_p_replacement.pdf}
\end{center}
%\vspace{-0.1in}
\caption{{\em sample automata: P automaton replaced}}
\label{automata_p_replaced}
\end{figure}


The intersection automaton will look after P automaton replacement Fig.\ref{graph_p_replaced}

\begin{figure}
\begin{center}
\includegraphics[width=50mm]{graph_p_replaced.eps}
\end{center}
%\vspace{-0.1in}
\caption{{\em Intersection automata of $P, Q and R$ after $P$ got replaced}}
\label{graph_p_replaced}
\end{figure}

After Q automaton replacement the control loops will look like Fig. \ref{automata_q_replaced}

\begin{figure}
\begin{center}
\includegraphics[width=50mm]{graph_after_q_replacement.pdf}
\end{center}
%\vspace{-0.1in}
\caption{{\em  sample automata: Q automaton replaced}}
\label{automata_q_replaced}
\end{figure}



The intersection automaton will look after Q automaton replacement Fig.\ref{graph_q_replaced}

 \begin{figure}
\begin{center}
\includegraphics[width=50mm]{graph_q_replaced.eps}
\end{center}
%\vspace{-0.1in}
\caption{{\em Intersection automata of $P, Q and R$ after $Q$ got replaced}}
\label{graph_q_replaced}
\end{figure}


After R automaton replacement the control loops will look like Fig. \ref{automata_r_replaced}

\begin{figure}
\begin{center}
\includegraphics[width=50mm]{graph_after_r_replacement.pdf}
\end{center}
%\vspace{-0.1in}
\caption{{\em sample automata: R automaton replaced}}
\label{automata_r_replaced}
\end{figure}


The intersection automaton will look after R automaton replacement Fig.\ref{graph_r_replaced}

  \begin{figure}
\begin{center}
\includegraphics[width=50mm]{graph_r_replaced.eps}
\end{center}
%\vspace{-0.1in}
\caption{{\em Intersection automata of $P, Q and R$ after $R$ got replaced}}
\label{graph_r_replaced}
\end{figure}