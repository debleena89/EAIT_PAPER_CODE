\section{Problem statement and Motivating Example} \label{sec3}
\noindent

\textcolor{red}{To explain our problem we are considerig a system consisting of many control
applications. Each of these control applications is partitioned into a set of
tasks. Each of these tasks are mapped onto different processors. These tasks 
of the controllers communicates via shared buses. The tasks on each processors
are scheduling using a given scheduling policy. The messages on the shared buses
are also scheduled using a given arbitration policy.}

\textcolor{red}{ will use the term platform to refer to such an (i) architecture (ii) the 
different scheduling and arbitration policies used. }

\textcolor{red}{ let us consider two control applications control loop1 and control loop2.
Control loop1 performs two tasks $T_1$ and $T_2$, where $T_1$ and $T_2$ mapped onto 
different processors. $T_1$ for example can be a sensor task and $T_2$ can be a controller 
task. $T_1$ and $T_2$ communicate via a message stream M, consisting of an infinite sequence 
of message instances $m_1$, $m_2$, .... $T_3$ is another task from control loop2. Assuming 
that $T_1$, $T_2$, and $T_3$ senses at a constant periodic rate. The two control loops are 
reresented by finite state diagram as in Fig.1 }
 

\begin{figure}[h]
\begin{center}
\includegraphics[width=3.0in]{diagram.pdf}
\end{center}
\vspace{-0.1in}
\caption{{\em Control Loops sharing bus}}
\label{fig:automaton}
\end{figure}

In the above diagram, the controllers are represented by state diagram. 
There are 6 states in control loop1, among them only $q_2$ 
and $q_3$ are sharing the bus. In control loop2, there are 4 states, and among them
only one state $p_0$ is sharing the bus. So this common bus will be shared by only
these three states. All these three states will access the bus channel in Time-Division 
multiple access method under certain schedulable constraint. 

The bus access by these states can be thought of as an infinite tree, whose nodes 
are the states of the controlling automata which are sharing the bus and
each path of the tree gives a particular scheduling sequence. To model the problem,
we have considered a finite length tree say, length of the tree is 3. The leaf node of the tree
gives the possible scheduling sequences with these states.

\begin{figure}[h]
\begin{center}
\includegraphics[width=3.5in]{tree_automata.pdf}
\end{center}
\vspace{-0.1in}
\caption{{\em Tree automata for scheduling sequence }}
\label{fig:automaton}
\end{figure}

The scheduling sequence construct the states for the finite state machine. After adding
each participating states with the states of finite state machine, we want to get a cycle from
thses states. After getting the cycles we found that all of the participating states are
present in the omega regular expression extracted from this cycle.

\begin{figure}[h]
\begin{center}
\includegraphics[width=1.5in]{cycle_1.pdf}
\end{center}
\vspace{-0.1in}
\caption{{\em Cycle construction on FSM }}
\label{fig:automaton}
\end{figure}


The obtained omega Regular expression from the above transition is $(p_0q_2q_3)^\omega$.

Let us consider that control loop1 is replaced, $q_2$ and $q_3$ are not sharing the bus now.

\begin{figure}[h]
\begin{center}
\includegraphics[width=3.0in]{diagram_replaced.pdf}
\end{center}
\vspace{-0.1in}
\caption{{\em Control loop 1 got modified }}
\label{fig:automaton}
\end{figure}

In the above diagram, $q_2$ and $q_3$ from control loop1 are not sharing the bus, only $p_0$ is 
sharing the bus from control loop2. The tree automata only for $p_0$  state will look like:

\begin{figure}
\begin{center}
\includegraphics[width=5mm]{diagram_single_state.pdf}
\end{center}
\vspace{-0.1in}
\caption{{\em Tree automata for scheduling after control loop 1 got modified }}
\label{fig:automaton}
\end{figure}

In the above diagram, $q_2$ and $q_3$ from control loop1 are not sharing the bus, only
$p_0$ is sharing the bus from control loop2. After constructing the tree automata for
single state as depicted in Fig.5, we get a single sceduling sequence $p_0p_0p_0$ .
Only one transition is possible here:
The omega Regular expression obtained from this transition is $(p_0)^\omega$ . In this
expression we obtain only a single state $p_0$ , belongs to control loop2. But there
are no states from control loop1. In this situation we can say that code replacement attack
has taken place.

\begin{figure}[h]
\begin{center}
\includegraphics[width=1.0in]{one_state.pdf}
\end{center}
\vspace{-0.1in}
\caption{{\em Cycle construction on one state of FSM}}
\label{fig:automaton}
\end{figure}



