\section{Problem statement and Motivating Example} \label{sec3}
\noindent
% A discrete time switched linear system is modeled by a finite set of real 
% $n \times n$ matrices $ \sum = \{A_i\}_{i \in I}$, $|I|$ $<$ $\infty $, as defined in~\cite{AlurW08}. 
% Infinite words $\sigma = I^\omega$ are used to describe control schedules. 
% Given a schedule $\sigma \in I^\omega$, and an initial state $x_0$ $\in$ 
% $\mathbb{R}^n$, the dynamics of the system is given by:
% 
% \begin{center}
%        \textcolor{black}{ $x_{k+1} = A_{\sigma_k} x_k$ }
% \end{center}
%\textbf{Problem definition of the problem:}
For a discrete-time  linear time invariant system as above, we assume we are given 
a set of $n$ controllers $ A = { {A_1}, {A_2}, ....{A_n}} $, each represented 
with the dynamics of its control actions specified with individual transition matrices. As discussed before,  
a control schedule is modeled as an interleaving of these control actions with the plant dynamics, as specified 
in the transition matrices. A control schedule is considered exponentially stable, if the composition of the plant and the controllers entail a non-empty language. 
As discussed in the previous section, the language of exponential stability is 
B\"{u}chi-recognizable. Formally, we are given the following:

\begin{itemize}
 \item We are given a plant ${\cal P}$, specified, as in Section~\ref{sec2}.
 \item We have  a set of $n$ control actions $A_1, A_2, \ldots A_n$ for $n$ controllers.
 \item A stability computation window length $L$ and a stability margin 
$\rho$ for $(L, \rho)$ exponential stability.
 \item A modified set of controllers $A'_1, A'_2, \ldots A'_n$, possibly resulting out of code replacement attacks. The two sets $A_1, A_2, \ldots A_n$ 
       and $A'_1, A'_2, \ldots A'_n$ differ in at least one $i$ (1 $\leq$ $i$ $\leq$ n).
%  \item For $L$ length scheduling sequence there will be possible $ n^L $ combinations.
%  \item Only exponentially stable sequence will be considered. Exponential stability$(L,\rho)$ 
%        computed for $L$ length sequence.
%  \item To construct a B\"{u}chi automaton with these stable states.   
\end{itemize}

\noindent
{\bf Objective:} 
Our objective is to develop an automatic framework for analyzing the effect of the modified set of controllers on the schedulability of the system as far as the 
exponential stability criterion is considered.
We assume that the original system is schedulable, and there indeed exists a scheduler that can ensure that the system resulting out of the composition of the 
plant and the original set of controllers is exponentially stable. 
 If the system resulting out of 
the combination of the plant and the modified set of controllers does not 
remain schedulable, we conclude that the system has been subject to a code replacement attack. We follow the same prescription for schedulability analysis, as given in~\cite{GhoshMDHD16}. %for schedulability analysis.    
%        If a B\"{u}chi Automaton cannot be constructed with these stable states, 
%        then it will be concluded that at least one of the control actions has been 
%        attacked, otherwise these stable states would \textcolor{red}{have Resulted} into a B\"{u}chi Automaton
%        which possess at least one cycle over the states.
The following example illustrates the philosophy. 
     
We consider a discrete-time switched linear system having
three control actions ${A_0, A_1, A_2}$ each specified by $2$ $\times$ $2$ matrices. Let us consider the stability computation window length is $4$, i.e., $\sigma = { \sigma_1, \sigma_2, 
\sigma_3, \sigma_4}$. So possible number of schedules for this scenario is $3^4$ = $81$.  
But all these schedules do not result into an exponentially stable state. 
% From all 
% possible states some states will be filtered out to guarantee exponential stability, 
% say $z$ number of states are stable among these $81$ states. Now, adding all possible
% transitions on these $z$ states will get a finite state machine. If a cycle can be 
% obtained from this state transitions, it will signify that this resulting automaton
% is a B\"{u}chi automaton and the string obtained from the cycle is $\omega$-regular. 
% A schedulable sequences of control actions is a $\omega$-regular string which 
% guarantees that the chosen set of schedules for the control actions.
% 
% 
% \noindent
% The above idea is captured to detect code replacement attack on such a 
% discrete-time switched linear system. In this work we have tried to illustrate an idea
% that if a control action is hijacked then the system will not remain schedulable, i.e.,  
% the states cannot form B\"{u}chi automaton.
Let us consider the following three control actions:

   $A_0 = ([2.0, -1.75],[2.0, -2.0])$
   
   $A_1 = ([0.25, 1.75],[0.25, -0.25])$

   $A_2 = ([0.5, 1.85],[0.5, -0.5])$
   
\noindent
Assuming a stability margin of 0.5, we can easily verify that among the possible $81$ states, only $14$ states are stable, and the automaton formed
with these states contains cycles. This means the control actions are schedulable.

%\vspace{-0.1 in}
\begin{figure}[h]
\begin{center}
\includegraphics[width=90mm, height=20mm]{schedulable_new.pdf}
\end{center}
%\vspace{-0.1in}
\caption{{\em Automaton construction with stable states}}
\label{fig:cycle found}
\end{figure}


\noindent
We now consider another set of control actions:
   
   $A_0 = ([2.0, -1.75],[2.0, -2.0])$
   
   $A_1 = ([1.65, -2.5],[1.66, 1.0])$

   $A_2 = ([0.5, 1.85],[0.5, -0.5])$
   
%\vspace{-0.1 in}
\begin{figure}

\begin{center}
\includegraphics[width=10mm, height = 10mm]{non_schedulable.pdf}
\end{center}
%\vspace{-0.1in}
\caption{{\em  Only one state found}}
\label{fig:Only one state}
\end{figure}

\noindent
In this case, among the possible $81$ states, only $1$ state is stable. A cycle cannot be obtained 
from this state, which results in a non-schedulable control action.
In the modified controller set, only $A_1$ has been changed from the previous 
set of control actions, the other 
two control actions remain the same. As we see from above, the loss of exponential stability can be a useful indicator for concluding that code replacement attacks have been carried out on a given system. This is the problem setting addressed in this paper.


% \subsection{Future Scope}
% \noindent
% In lieu of the scalability issue with model based assertion verification methods, dynamic assertion based verification techniques are inevitable to follow for software verification sign-off. Hence, we see to extend the proposed frame work to employ a mechanism for the generation of minimal subset of test suites that cover all the user specified assertions and subsequently use this framework to complete the testing process. 
%\vspace{-0.2 in}



