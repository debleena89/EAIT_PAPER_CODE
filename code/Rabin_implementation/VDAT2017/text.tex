\section{Problem statement and Motivating Example} \label{sec3}
\noindent
The motive of our work is to detect code replacement attack. Code replacement attack
signifies some changes to the system. We are trying to capture the results of the 
attack from the perspective of scheduling.  Here in this work, we are examining
how this changes to the system effects the schedulability of the system. 

To explain our problem we are considering a system consisting of many control
applications. Each of these control applications is partitioned into a set of
tasks. Each of these tasks are mapped onto different processors. These tasks 
of the controllers communicates via shared buses. The tasks on each processors
are scheduling using a given scheduling policy. The messages on the shared buses
are also scheduled using a given arbitration policy. When a process is executed by processor
it goes through a number of states before termination. Among all these states some
states require bus access. Those states are allocated to the bus when their term
comes. 

\begin{figure}[h]
\begin{center}
\includegraphics[width=2.0in]{system_diagram_3_loop.pdf}
\end{center}
\vspace{-0.1in}
\caption{{\em Control Loops sharing bus}}
\label{fig:automaton}
\end{figure}

The diagram depicted in Fig.1 comprising of three control loops $P,Q$ and $R$. 
Each control loop has two tasks $T_1$ and $T_2$. Here we are considering, from each of these tasks at least one
state requires bus access. Let us consider $p_0$ and $p_1$ from control application $P$,
$q_0$ and $q_1$ from control application $Q$ and $r_0$ and $r_1$ from control application
$R$ are the bus access states. The state diagram representation of these control loops is 
depicted in Fig. 2.

\begin{figure}
\begin{center}
\includegraphics[width=2.0in]{state_diagram_task.pdf}
\end{center}
\vspace{-0.1in}
\caption{{\em Automata reprenting Bus accessing states}}
\label{fig:automaton}
\end{figure}
 As we have stated earlier that we are trying to observe the effect 
of code replacement attack in terms of scheduling sequence, our system will be examinig the
scheduling sequence after a particular interval of time. Say, our expectation is that at
every point of time we want to obtain a scheduling sequence where at least one bus accessing 
state from each control application is present.
In this work such scheduling expression are represented by $\omega$-regular language. Extraction 
of such an expression for the automaton given in Fig. 2 is depicted in Fig. 3.

\begin{figure}[h]
\begin{center}
\includegraphics[width=3.0in]{state_chage_many.pdf}
\end{center}
\vspace{-0.1in}
\caption{{\em Cycle construction and $\omega$-regular expression extraction}}
\label{fig:automaton}
\end{figure}
From the cycle depicted in Fig.3 we can extract the $\omega$-regular expression 
$(p_1q_0q_1r_0r_1p_0p_0p_1)^\omega$.

Let us consider, if a code replacement attack take place on this system and one of the 
control applications has lost the bus access. As a result of this attack no state from
this finite state machine will take part in bus access that we can capture by checking
the scheduling sequence. The scheduling sequence will not contain any state of this replaced
control application as a consequence of the attack. 

\begin{figure}
\begin{center}
\includegraphics[width=2.0in]{state_diagram_task_replaced.pdf}
\end{center}
\vspace{-0.1in}
\caption{{\em Automata reprenting Bus accessing states after code replacement attack}}
\label{fig:automaton}
\end{figure}

Let us consider, control application $P$ got effected due to code replacement attack,
bus accessing states are not getting the bus access. The $\omega$-regular expression 
will be depicted in Fig.5 :

\begin{figure}[h]
\begin{center}
\includegraphics[width=3.0in]{state_chage_many.pdf}
\end{center}
\vspace{-0.1in}
\caption{{\em Cycle construction and $\omega$-regular expression extraction}}
\label{fig:automaton}
\end{figure}
From the cycle depicted in Fig.5 we can extract the $\omega$-regular expression 
$(r_0r_1q_1q_0q_0q_0q_0)^\omega$. The expressin does not contain any state from
the control application $P$. 

On the other hand, if it would happen 
that after code replacement attack instead of the earlier state some other state is accessing
the bus and the earlier ones (those who were accessing the bus before the attack took place)
lost their bus access then our expectation cannot capture that the attack has taken place.

